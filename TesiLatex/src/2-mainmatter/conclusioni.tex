\chapter{Conclusioni}





Il lavoro svolto ha riguardato lo studio delle botnet, sia dal punto di vista teorico che pratico. Sono state analizzate diverse implementazioni di botnet open source per comprenderne il funzionamento effettivo e testare la loro rilevazione attraverso NIDS e HIDS come Security Onion, Ossim e Wazuh. Grazie all'uso di agent installati sulle macchine da monitorare e strumenti di rilevazione di rete, è stato possibile identificare correttamente la presenza di bot all'interno dell'infrastruttura, confermando l'efficacia dei comuni strumenti di rilevazione basati su pattern matching di fingerprint.

Tuttavia, i test hanno anche evidenziato le attuali debolezze degli approcci fingerprint based, come le complicanze di rilevazione dovute a tecniche di detection evasion, sfruttate dalle moderne botnet in circolazione, o l'impossibilità di rilevare malware o attività zero day che sono prive di firme identificative. Questo ha portato alla considerazione di approcci di rilevazione delle botnet basati su anomalie, che rappresentano oggi la principale classe di tecniche di rilevazione di botnet studiate dai ricercatori.

All'interno del lavoro svolto, sebbene non siano stati inseriti all'interno del documento, vi è anche lo studio di strumenti di automazione come Ansible \cite{ansible} e Terraform \cite{terraform} che possono essere utilizzati per automatizzare il processo di deploy  e la configurazione dell'infrastruttura di testing per possibili estensioni future.

In conclusione, il lavoro svolto ha permesso di comprendere meglio il funzionamento delle botnet e di valutare l'efficacia degli strumenti di rilevazione utilizzati. Tuttavia, data l'evoluzione continua delle tecniche di evasion delle botnet, è importante continuare a studiare e sviluppare nuovi approcci di rilevazione per garantire la sicurezza delle infrastrutture IT.