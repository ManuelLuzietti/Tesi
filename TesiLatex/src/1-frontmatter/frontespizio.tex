% !TeX root = ../../tesi.tex
% !TeX encoding = UTF-8 Unicode
% !TeX spellcheck = it_IT

\begin{Preambolo*}
  \usepackage{fontspec}
  \setmainfont[Ligatures=TeX]{Latin Modern Roman}
\end{Preambolo*}
\begin{frontespizio}
  \Universita{Bologna}        % aggiunge da sé “Università degli Studi di”.
  \Istituzione{%
    Alma Mater Studiorum --- Università di Bologna \\%
    Campus di Cesena%
  }
  \Divisione{Dipartimento di Informatica --- Scienza e Ingegneria}
  \Corso[Laurea triennale]{Ingegneria e Scienze Informatiche}
  \Annoaccademico{2021--2022}
  \Titolo{Studio delle Botnet per mezzo di IDS}
  \Sottotitolo{Tesi di Laura }
  %\Preambolo{\renewcommand{\frontsmallfont}[1]{\small}}       % non viene stampata la matricola
  \Preambolo{\renewcommand{\frontsmallfont}[1]{\small Matr.}} % abbrevia la matricola
  \Candidato[0000937684]{Manuel~Luzietti}
  \NCandidato{Presentata da}  % sostituisce la parola “Candidato”
  \Relatore{Prof.~G~D'Angelo}
  \Correlatore{Prof.~C~Barbone}
  \Piede{%                    % sostituisce la scritta “Anno Accademico” nel piede
    %III sessione di laurea \\%
    Anno Accademico 2021--2022%
  }
\end{frontespizio}

% Necessario per Overleaf: compila il TeX del frontespizio subito dopo averlo generato
\IfFileExists{\jobname-frn.pdf}{}{%
\immediate\write18{lualatex \jobname-frn}}
