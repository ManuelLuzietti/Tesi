% !TeX root = ../../tesi.tex
% !TeX encoding = UTF-8 Unicode
% !TeX spellcheck = it_IT

\chapter{Introduzione}
Il rapporto dell'Associazione Italiana per la Sicurezza Informatica \cite{Clusit} ha rivelato che nel 2022 sono state individuate ben 7.71 miliardi di minacce \textit{Botnet} in tutto il mondo, mentre gli attacchi \textit{Distributed Denial of Service} (DDoS)\footnote{Attacco informatico in cui si cerca di esaurire le risorse computazionali della vittima attraverso molteplici richieste provenienti da plurime sorgenti} in Italia, spesso veicolati attraverso Botnet, sono raddoppiati rispetto all'anno precedente. Dei 6.32 miliardi di \textit{malware} rilevati, 39.88 milioni solo in Italia, il rapporto ha evidenziato una particolare polarizzazione rispetto al 2021 attorno a \textit{Zeus}, \textit{Ramnit} e \textit{QakBot}, tutti in grado di costringere la macchina infetta in una rete di \textit{bot} \cite{FSecureLabsBlogRamnit,CisaQakBot,KaskerkyZeus,IBMSecBlogRamnit}. Inoltre, le rilevazioni effettuate dai device appartenenti all’Autonomous System (AS) di Fastweb mostrano che il 3\% delle infezioni è causato da \textit{Conficker} (più di 9 milioni di macchine infette stimate \cite{FSecureLabsBlogConfickersData}), con al secondo posto \textit{Avalanche-Andromeda} (responsabile della distribuzione di circa 80 famiglie di malware), entrambi attribuibili alla classe Botnet.

Questi sono solo una parte dei dati allarmanti che dimostrano quanto le Botnet siano diffuse e permeate nel mondo del \textit{cybercrimine}. Per questo motivo rimangono tutt’ora un importante oggetto di studio da parte di provider di sistemi di sicurezza ICT e ricercatori, all’interno di un contesto in cui l’Italia durante l’anno passato ha registrato il numero più alto di incidenti informatici mai registrato\footnote{2.489 incidenti di cui almeno 1500 con severità compresa tra alta e critica}.

Le Botnet sono reti costituite da macchine infette, dette bot, che eseguono comandi provenienti da una o più macchine, detti Command and Control Server (C\&C Server), impartiti dal botmaster (controllore dei bot). Il concetto alla base di questa tecnologia è così semplice da rendere labile il confine con altri malware dalla struttura simile. Si pensi ad esempio a un ipotetico \textit{Trojan}\footnote{Virus informatico nascosto all'interno di un altro applicativo}, capace di inviare le informazioni rubate dagli utenti verso server di raccolta dati; questo ne condividerebbe la possibile struttura, seppure non fosse in grado di eseguire comandi. Non è una novità infatti che un malware inizialmente progettato per uno scopo preciso e limitato, venga poi aggiornato in una versione capace di eseguire comandi, trasformandolo in tutto e per tutto in una Botnet, estendendone le capacità col minimo sforzo (ne è un esempio \textit{Ramnit} \cite{IBMSecBlogRamnit}). Queste reti possono essere utilizzate per una vasta gamma di attività illecite, tra cui  Remote Administration Tool (RAT)\footnote{Virus informatico capace di permettere accesso attraverso la rete a macchina vittima}, DDoS, distribuzione di malware, furto di informazioni personali, \textit{spam}, etc.

L’elaborato che segue si pone come obiettivo lo sullo studio di queste tecnologie, partendo da un punto di vista teorico per poi procedere con l'analisi di due Botnet \textit{open source}\footnote{con Botnet open source si intende sviluppata a scopi divulgativi/didattici, con codice sorgente visualizzabile e a cui è possibile contribuire.} in un ambiente controllato appositamente strutturato. La trattazione della tesi sarà organizzata in modo da studiarne in primo luogo le topologie, i protocolli e le tecniche di offuscamento utilizzate dalle Botnet. Successivamente, verranno analizzati il \textit{deployment} dell'infrastruttura e gli strumenti utilizzati, inclusi gli  \textit{intrusion detection system} installati per il testing. Verrà quindi presentata una sezione inerente gli Agent\footnote{Programma attivo in background che raccoglie informazione ed esegue task su macchina monitorata}, prima di proseguire con l’analisi della seconda rete di bot. Infine, verranno analizzati gli approcci di rilevazione utilizzati e considerati, per poi passare a una breve trattazione di altri possibili approcci.
